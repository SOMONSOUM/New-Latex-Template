\chapter{គោលការណ៍តម្រួតនៃរលក និងរលកជញ្រ្ជំ}
\quad មុនយើងនឹងសិក្សាលម្អិតអំពីមេរៀននេះយើងត្រូវស្វែងយល់អំពីនិយមន័យនៃពាក្យសំខាន់ៗដែលត្រូវប្រើក្នុងមេរៀននេះ មានដូចជា៖
\begin{definition}
	\begin{enumerate}
		\item \emph{\kml រលក{\en(Waves)}} ជាដំណាលរញ្ជួយរបស់ម៉ូលេគុល ឬអង្គធាតុក្នុងមជ្ឈដ្ឋានមួយ។
		\item \emph{\kml ជំហានរលក{\en(Wave Length)}} ជាចម្ងាយដែលរលកដាលបានក្នុងរយៈពេលមួយខួប។
		\item \emph{\kml ខួប{\en(Period)}} ជារយៈពេលដែលរលកដាលបានមួយលំយោលពេញ។
		\item \emph{\kml ប្រេកង់{\en(Frequency)}} ជាចំនួនលំយោលដែលធ្វើបានក្នុងមួយវិនាទី។
	\end{enumerate}
\end{definition}
\section{គោលការណ៍តម្រួតនៃរលក{\en(Superposition Principle)}}
\begin{definition}
	កាលណារលកពីរ ឬច្រើនដាលឆ្លងកាត់មជ្ឈដ្ឋានតែមួយបម្លាស់ទីសរុបនៃរាល់ចំណុចណាក៏ដោយរបស់រលក ស្មើនឹងផលបូកវ៉ិចទ័រនៃបណ្តាលចំណុចបម្លាស់ទីរបស់រលកទោលទាំងនោះ។\\ រលកបែបនេះហៅថា \textbf{រលកលីនេអ៊ែរ ឬរលកតម្រួត}។ យើងអាចនិយាយជារួមមកបានថា រលកតម្រួត ឬរលកលីនេអ៊ែ ជាផលបូកនៃរលកពីរ ឬច្រើនដែលដាលឆ្លងកាត់មជ្ឈដ្ខានតែមួយ។
\end{definition}
\subsection{សមីការចលនាស៊ីនុយសូអុីនៃរលក(ត្រង់ប្រភព)}
\section{សំណួរ លំហាត់អនុវត្តន៍ និងកិច្ចការផ្ទះ}